\documentclass[12pt]{amsart}
\usepackage{amsmath, amssymb, amsfonts, amsthm, fullpage, stmaryrd,  url, hyperref}
\usepackage[all]{xy}

\newtheorem{theorem}{Theorem}[section]
\newtheorem{lemma}[theorem]{Lemma}
\newtheorem{cor}[theorem]{Corollary}
\newtheorem{prop}[theorem]{Proposition}

\theoremstyle{definition}
\newtheorem{defn}[theorem]{Definition}
\newtheorem{example}[theorem]{Example}
\newtheorem{exercise}[theorem]{Exercise}
\newtheorem{hypothesis}[theorem]{Hypothesis}
\newtheorem{convention}[theorem]{Convention}
\newtheorem{remark}[theorem]{Remark}

\numberwithin{equation}{theorem}

\newcommand{\bA}{\mathbf{A}}
\newcommand{\be}{\mathbf{e}}
\newcommand{\bv}{\mathbf{v}}

\newcommand{\CC}{\mathbb{C}}
\newcommand{\Fp}{\mathbb{F}_p}
\newcommand{\Qp}{\mathbb{Q}_p}
\newcommand{\QQ}{\mathbb{Q}}
\newcommand{\RR}{\mathbb{R}}
\newcommand{\Zp}{\mathbb{Z}_p}
\newcommand{\ZZ}{\mathbb{Z}}
\newcommand{\calO}{\mathcal{O}}
\newcommand{\calR}{\mathcal{R}}

\newcommand{\frakm}{\mathfrak{m}}
\newcommand{\frako}{\mathfrak{o}}

\newcommand{\dual}{\vee}

\DeclareMathOperator{\BPair}{\mathbf{BPair}}
\DeclareMathOperator{\Dfm}{\mathbf{Dfm}}
\DeclareMathOperator{\FEt}{\mathbf{F\acute{E}t}}
\DeclareMathOperator{\Frac}{Frac}
\DeclareMathOperator{\Gal}{Gal}
\DeclareMathOperator{\GL}{GL}
\DeclareMathOperator{\Hom}{Hom}
\DeclareMathOperator{\Ind}{Ind}
\DeclareMathOperator{\Maxspec}{Maxspec}
\DeclareMathOperator{\perf}{perf}
\DeclareMathOperator{\PhiGamma}{\mathbf{\Phi \Gamma}}
\DeclareMathOperator{\PhiGammatilde}{\widetilde{\mathbf{\Phi \Gamma}}}
\DeclareMathOperator{\PhiMod}{\mathbf{\Phi Mod}}
\DeclareMathOperator{\Proj}{Proj}
\DeclareMathOperator{\rank}{rank}
\DeclareMathOperator{\Rep}{\mathbf{Rep}}
\DeclareMathOperator{\Res}{Res}
\DeclareMathOperator{\Spec}{Spec}
\DeclareMathOperator{\Trace}{Trace}
\DeclareMathOperator{\triv}{triv}
\DeclareMathOperator{\VB}{\mathbf{VB}}

\begin{document}

\title{On categories of $(\varphi, \Gamma)$-modules}
\author{Kiran S. Kedlaya and Jonathan Pottharst}
\thanks{Supported by NSF (grant DMS-1501214), UC San Diego (Warschawski Professorship).}
\date{\textit{unstable draft}; version of August 15, 2015}

\begin{abstract}
Let $K$ be a complete discretely valued field of mixed characteristics $(0,p)$ with perfect residue field. One of the central objects of study in $p$-adic Hodge theory is the cate\-gory of continuous representations of the absolute Galois group of $K$ on finite-dimensional $\QQ_p$-vector spaces. In recent years, it has become clear that this category can be studied more effectively by embedding it into a larger category of \emph{$(\varphi, \Gamma)$-modules}; this larger category plays a role analogous to that played by the category of vector bundles on a compact Riemann surface in the Narasimhan-Seshadri theorem on unitary representations of the fundamental group of said surface. This category turns out to have a number of distinct natural descriptions, which on one hand suggests the naturality of the construction, but on the other hand forces one to use different descriptions for different applications. We provide several of these descriptions and indicate how to translate certain key constructions, which were originally given in the context of modules over power series rings, to the more modern context of perfectoid algebras and spaces.
\end{abstract}

\maketitle

Throughout, let $p$ be a prime number and let $K$ be a \emph{$p$-adic field}, by which we mean a complete discretely valued field of mixed characteristics $(0,p)$ and perfect (but not necessarily finite) residue field. In $p$-adic Hodge theory, one studies the relationship between different cohomology theories associated to algebraic (and more recently analytic) varieties over $K$. One of these theories is \'etale cohomology with $\QQ_p$-coefficients, which naturally leads to a detailed study of continuous representations of $G_K$, the absolute Galois group of $K$, acting on finite-dimensional $\QQ_p$-vector spaces. In recent years, it has been discovered that this category embeds naturally and usefully into a larger category of \emph{$(\varphi, \Gamma)$-modules}, in much the same way that the Narasimhan-Seshadri theorem embeds the category of irreducible unitary representations of the fundamental group of a compact Riemann surface into the category of vector bundles on that surface.

One indication of the naturality of the category of $(\varphi, \Gamma)$-modules is that it can be given several equivalent descriptions. The original description emerging out of the work of Fontaine, Colmez, Berger, and others involves modules over certain rings of convergent Laurent series. In this description, a privileged role is played by the infinite extension $K_\infty$ of $K$ obtained by adjoining all $p$-power roots of unity; for example, the base rings carry an action of the Galois group of $K_\infty$ over $K$ (the eponymous $\Gamma$). However, several other descriptions of the same category can now be given in which the cyclotomic tower plays no distinguished role; these include the description by Berger in the language of $B$-pairs and the description by Fargues--Fontaine in terms of vector bundles on certain one-dimensional noetherian schemes. The latter description arises very naturally within the geometric reinterpretation of $p$-adic Hodge theory in the language of \emph{perfectoid spaces}, as in the work of Scholze and Kedlaya-Liu.

It is thus natural to ask to what extent constructions made in the language of $(\varphi, \Gamma)$-modules, particularly those which have become relevant in Iwasawa theory, can be realized using these alternate descriptions. The purpose of this paper is to give some answers to this question, using the theory of arithmetic deformations of $(\varphi, \Gamma)$-modules developed in \cite{kpx}. The basic idea is to exchange the explicit use of the cyclotomic tower in the classical construction of $(\varphi, \Gamma)$-modules for an arithmetic deformation parametrizing cyclotomic twists. This makes it natural to consider other deformations corresponding to other $p$-adic Lie groups, including nonabelian ones.

Before concluding this introduction, we set a few running notations. Our primary model for these and other notations is \cite{kpx}.
\setcounter{theorem}{0}
\begin{hypothesis}
Throughout this paper, as in this introduction, let $K$ be a complete discretely valued field of mixed characteristics with perfect residue field $k$; we do not assume $k$ is finite (i.e., that $K$ is a finite extension of $\QQ_p$) unless explicitly specified.
Put $F = W(k)[1/p]$ for $W(k)$ the ring of Witt vectors over $k$, so that $K/F$ is a finite totally ramified extension.
Let $A$ be an affinoid algebra over $\QQ_p$ in the sense of Tate, rather than the more expansive sense of Berkovich. 
\end{hypothesis}

\section{The original category of $(\varphi, \Gamma)$-modules}
\label{sec:categories}

We begin by describing the original construction of the category of $(\varphi, \Gamma)$-modules,
into which the category of continuous representations of $G_K$ on finite-dimensional $\QQ_p$-vector spaces embeds; this is most explicit in the case $K=F$.
In preparation for our later discussion, we escalate the level of generality to accommodate representations valued in affinoid algebras.

\begin{defn}
Let $\Rep_A(G_K)$ denote the category of representations of $G_K$ (the absolute Galois group of $K$) on finite projective $A$-modules.
\end{defn}

\begin{defn}
Let $\calR^\infty_{F,A}$ be the ring of rigid analytic functions on the disc $\left| \pi \right| < 1$ over $F \widehat{\otimes}_{\QQ_p} A$. This ring is complete for the topology of uniform convergence on quasicompact subspaces (Fr\'echet topology). The ring admits a continuous endomorphism $\varphi$ defined by the formula
\begin{equation} \label{eq:phi formula}
\varphi \left( \sum_n c_n \pi^n \right) =  \sum_n \varphi_F(c_n) ((1 + \pi)^p-1)^n,
\end{equation}
where $\varphi_F$ denotes the $A$-linear extension of Witt vector Frobenius map on $F$.
The group $\Gamma = \ZZ_p^\times$ also admits a continuous action on $\calR^\infty_{F,A}$ defined by the formula
\begin{equation} \label{eq:gamma formula}
\gamma \left( \sum_n c_n \pi^n \right) = \sum_n c_n ((1 + \pi)^\gamma-1)^n \qquad (\gamma \in \Gamma)
\end{equation}
interpreting $(1 + \pi)^\gamma$ as the binomial series 
\[
(1 + \pi)^\gamma = \sum_{n=0}^\infty \frac{\gamma(\gamma-1)\cdots (\gamma-n+1)}{n!} \pi^n.
\]
Note that the actions of $\varphi$ and $\Gamma$ commute.
\end{defn}

\begin{defn}
Let $\calR_{F,A}$ be the direct limit of the rings of rigid analytic functions on the annuli $* < \left| \pi \right| < 1$ over $F \widehat{\otimes}_{\QQ_p} A$. This ring is complete for the locally convex direct limit topology induced by the topologies of uniform convergence on quasicompact subspaces (the \emph{LF topology}). We extend the actions of $\varphi$ and $\Gamma$ on $\calR^\infty_{F,A}$ to continuous actions on $\calR_{F,A}$ using the same formulas \eqref{eq:phi formula}, \eqref{eq:gamma formula}.
\end{defn}

\begin{defn}
A \emph{$(\varphi, \Gamma)$-module} over $\calR_{F,A}$ is a finite projective $\calR_{F,A}$-module $M$ equipped with commuting semilinear actions of $\varphi$ and $\Gamma$ such that the action of $\Gamma$ is continuous for the LF topology. Let $\PhiGamma_{F,A}$ denote the category of such objects, viewed as an exact tensor category
with rank function $\rank_F: \Phi\Gamma_{F,A} \to \ZZ$ computing the rank of the underlying $\calR_{F,A}$-module.
\end{defn}

We will establish the following result in \S\ref{sec:alternate}.
\begin{theorem} \label{T:phi gamma embedding1}
There exists a full embedding $\Rep_A(G_F) \to \PhiGamma_{F,A}$.
\end{theorem}
In the interim, let us see how this result can be used to define a corresponding category with $F$ replaced by $K$.

\begin{defn}
Let $\calR_{K,A} \in \PhiGamma_{F,A}$ be the object of rank $[F:K]$ corresponding to $\Ind^{G_F}_{G_K} \rho_{\triv}$ via Theorem~\ref{T:phi gamma embedding}. The canonical isomorphisms $\rho_{\triv} \otimes \rho_{\triv} \cong \rho_{\triv}^\dual \otimes \rho_{\triv} \cong \rho_{\triv}$
then correspond to an associative morphism $\mu_K: \calR_{K,A} \otimes \calR_{K,A} \to \calR_{K,A}$; this gives $\calR_{K,A}$ the structure of a finite flat $\calR_{F,A}$-algebra equipped with continuous actions of $\varphi$ and $\Gamma$.

Let $\PhiGamma_{K,A}$ be the category of pairs $(M, \mu)$ for which $M \in \PhiGamma_{F,A}$
and $\mu: \calR_{K,A} \otimes M \to M$ is a morphism which is associative with respect to $\mu_K$, i.e., the compositions
\begin{gather*}
\calR_{K,A} \otimes \calR_{K,A}  \otimes M \stackrel{\mu_K \otimes 1}{\to} M_K \otimes M \stackrel{\mu}{\to} M, \\
\calR_{K,A} \otimes \calR_{K,A} \otimes M \stackrel{1 \otimes \mu}{\to} \calR_{K,A} \otimes  M \stackrel{\mu}{\to} M
\end{gather*}
coincide. 
In other words, these are finite projective $\calR_{K,A}$-modules equipped with commuting semilinear continuous actions of $\varphi$ and $\Gamma$.
We again view $\PhiGamma_{K,A}$ as an exact tensor category with rank function $\rank_K = \rank_F / [K:F]$ computing the rank of the underlying $\calR_{K,A}$-module.

Let $K'$ be a finite extension of $K$. Define the induction functor
$\Ind: \PhiGamma_{K',A} \to \PhiGamma_{K,A}$ 
and the restriction functor $\Res: \PhiGamma_{K,A} \to \PhiGamma_{K',A}$
by restriction of scalars and extension of scalars, respectively (sic), along the natural map $\calR_{K,A}  \to \calR_{K',A}$.
\end{defn}

We may then formally promote Theorem~\ref{T:phi gamma embedding1} as follows.
\begin{theorem} \label{T:phi gamma embedding}
There exists a full embedding $\Rep_A(G_K) \to \PhiGamma_{K,A}$ compatible with induction and restriction on both sides.
\end{theorem}

\begin{remark} \label{R:not connected}
The description of $\PhiGamma_{K,A}$ given above is consistent with \cite{kedlaya-new-phigamma} but not with most older references. The reason is that even if $A$ is connecetd, the ring $\calR_{K,A}$ is in general not connected; it is more typical to replace it with one of its connected components, and to replace $\Gamma$ with the stabilizer of that component.
See Remark~\ref{R:not connected2} for further discussion.
\end{remark}

\section{Interlude on perfectoid fields}
\label{sec:perfectoid}

In preparation for giving alternate descriptions of the category $\PhiGamma_{K,A}$, we introduce the basic theory of \emph{perfectoid fields}, which subsumes the earlier theory of \emph{norm fields} on which the classical theory of $(\varphi, \Gamma)$-modules is built. See \cite{kedlaya-new-phigamma} for more historical discussion.

\begin{defn}
Let $L$ be a field containing $K$ which is complete with respect to a nonarchimedean absolute value, denoted $\left| \bullet \right|$. Let $\frako_L$ denote the valuation subring of $L$ (i.e., elements of norm at most $1$). We say $L$ is \emph{perfectoid} if $L$ is not discretely valued and the Frobenius map on $\frako_L/(p)$ is surjective.
\end{defn}

\begin{example}  \label{exa:cyclotomic}
Let $L$ be the completion of $F(\mu_{p^\infty})$.
Then 
\begin{align*}
\frako_L &\cong (W(k)[\zeta_p, \zeta_{p^2}, \dots]/(1 + \zeta_p + \cdots + \zeta_p^{p-1}, \zeta_p - \zeta_{p^2}^p, \zeta_{p^2} - \zeta_{p^3}^p, \dots)^{\wedge}_{(p)} \\
\frako_L/(p) &\cong k[T_1, T_2, \dots]/(1+T_1 + \cdots + T_1^{p-1}, T_1 - T_2^p, T_2 - T_3^p, \dots),
\end{align*}
so $L$ is perfectoid.
\end{example}

For the remainder of \S\ref{sec:perfectoid}, let $L$ be a perfectoid field.

\begin{theorem} \label{T:perfectoid}
Define the multiplicative monoids
\[
\frako_{L^{\flat}} = \varprojlim_{x \mapsto x^p} \frako_L, \qquad
L^{\flat} = \varprojlim_{x \mapsto x^p} L.
\]
\begin{enumerate}
\item[(a)]
There is a unique way to promote $\frako_{L^{\flat}}$ and $L^{\flat}$ to rings
in such a way that the maps $\frako_{L^{\flat}} \to L^{\flat}$ and $\frako_{L^{\flat}} \to \frako_L \to \frako_L/(p)$ become ring homomorphisms.
\item[(b)]
The ring $L^{\flat}$ is a perfect field. In addition,
the function $L^{\flat} \to L \stackrel{\left| \bullet \right|}{\to} \RR$ is an absolute value with respect to which $L^{\flat}$ is complete with valuation subring $\frako_{L^{\flat}}$.
\item[(c)]
Any finite extension of $L$, equipped with the unique extension of the absolute value, is again perfectoid.
\item[(d)]
The functor $L' \mapsto L^{\prime \flat}$ defines an equivalence of categories between finite extensions of $L$ and $L^{\flat}$, and thereby a canonical isomorphism $G_L \cong G_{L^{\flat}}$.
\end{enumerate}
\end{theorem}
\begin{proof}
See \cite[\S 1]{kedlaya-new-phigamma} and references therein.
\end{proof}

\begin{defn}
Let $W(L)^b$...
\textbf{extended Robba ring, its topology}

Let $\PhiMod_{L,A}$ be the category of finite projective $\tilde{\calR}_{L,A}$-modules equipped with semilinear $\varphi$-actions.
\end{defn}

\begin{lemma}
For any $r,s$ with $0 < s \leq r/p$, we have
\[
\ker(\varphi-1: \tilde{\calR}^{[s,r]}_{L,A} \to \tilde{\calR}^{[s,r/p]}_{L,A}) = A.
\]
In particular, we have $\tilde{\calR}_{L,A}^{\varphi} = A$.
\end{lemma}
\begin{proof}
In the case $A = \QQ_p$, this is \cite[Corollary~5.2.4]{kedlaya-liu1}. 
The general case may be deduced by constructing a Schauder basis for $A$ over $\QQ_p$; see \cite[Lemma~2.2.9(b)]{kedlaya-liu1}.
\end{proof}

\begin{remark}
It is shown in \cite{kedlaya-noetherian} that the rings $\tilde{\calR}^{[s,r]}_{L,\QQ_p}$ are \emph{strongly noetherian}, that is, any affinoid algebra over such a ring is noetherian. In particular, the rings $\tilde{\calR}^{[s,r]}_{L,A}$ are strongly noetherian.
\end{remark}

\begin{theorem}
Let $L$ be a perfectoid field. Let $L'$ be the completion of a (possibly infinite) Galois algebraic extension of $L$ with Galois group $G$. Then the functor from
$\PhiMod_{L,A}$ to the category of objects of $\PhiMod_{L',A}$ equipped with continuous semilinear $G$-actions is an equivalence of categories.
\end{theorem}
\begin{proof}
\end{proof}



...

\section{Alternate descriptions of \texorpdfstring{$(\varphi, \Gamma)$}{(phi, Gamma)}-modules}
\label{sec:alternate}

We now give several alternate descriptions of the category $\PhiGamma_{K,A}$. In the process, we will establish Theorem~\ref{T:phi gamma embedding}; this will illustrate that cyclotomic extensions play a special role in the original construction, but not in some of
the alternate descriptions.

\begin{cor}
Let $L$ be a perfectoid field.
Let $\CC_L$ be a completed algebraic closure of $L$. Then the formula 
\[
V \mapsto (V \otimes_A \tilde{\calR}_{\CC_L, A})^{G_L}
\]
defines a full embedding $\Rep_A(G_L) \to \PhiMod_{L,A}$.
\end{cor}

\begin{defn}
Let $L$ be the completion of $K(\mu_{p^\infty})$; it is a perfectoid field by
Example~\ref{exa:cyclotomic} and Theorem~\ref{T:perfectoid}.
Let $\PhiGammatilde_{K,A}$ denote the category of finite projective $\tilde{\calR}_{L,A}$-modules equipped with continuous semilinear commuting actions of $\varphi$ and $\Gamma$.
\end{defn}

\textbf{some results}

\begin{defn}
\textbf{B-pairs}
Let $\BPair_{K,A}$ be the category of...
\end{defn}

\begin{defn}
Let $L$ be a perfectoid field. Define the graded ring
\[
P_{L,A} = \bigoplus_{n=0}^\infty P_{L,A,n}, \qquad P_{L,A,n} = \tilde{\calR}_{L,A}^{\varphi=p^n}.
\]
Let $\VB_{K,A}$ be the category of vector bundles on the scheme $\Proj P_{L,A}$.
\end{defn}

\begin{theorem}
The categories $\PhiGamma_{K,A}, \PhiGammatilde_{K,A}, \BPair_{K,A}, \VB_{K,A}$ are canonically equivalent (via functors to be described below).
\end{theorem}
\begin{proof}
\end{proof}

As a corollary, we may now establish Theorem~\ref{T:phi gamma embedding1}.
\begin{proof}[Proof of Theorem~\ref{T:phi gamma embedding1}]
\end{proof}

\begin{remark} \label{R:not connected2}
Following up on Remark~\ref{R:not connected}, we observe that for $L$ be the completion of $K(\mu_{p^\infty})$, the connected components of $\calR_{K,A}$ and $\tilde{\calR}_{K,A}$ coincide. In particular, the connected components of $\calR_{K,\QQ_p}$ correspond canonically to the connected components of $K \otimes_F F(\mu_{p^\infty})$; the group $\Gamma$ acting on $\calR_{K,\QQ_p}$ may be canonically identified with $\Gal(F(\mu_{p^\infty})/F)$; and the stabilizer in $\Gamma$ of any connected component of $K \otimes_F F(\mu_{p^\infty})$ is
the open subgroup $\Gamma_K = \Gal(K(\mu_{p^\infty})/K)$ of $\Gamma$. In the prior literature (excluding \cite{kedlaya-new-phigamma}), it is customary to replace $\calR_{K,\QQ_p}$ with a connected component and then replace $\Gamma$ with $\Gamma_K$; this gives category of $(\varphi, \Gamma_K)$-modules equivalent to our category of $(\varphi, \Gamma)$-modules. See \cite[Remark~2.2.12]{kedlaya-new-phigamma} for further discussion.
\end{remark}


\section{Slopes of $(\varphi, \Gamma)$-modules}

We now introduce the important concept of \emph{slopes} of $(\varphi, \Gamma)$-modules. The basic theory is motivated by the corresponding theory of slopes of vector bundles on algebraic varieties (especially curves). In the process, we identify the essential image of the embedding functor of Theorem~\ref{T:phi gamma embedding1} in case $A$ is a field.


\textbf{degree, slope}

\begin{defn}
Throughout this definition, suppose that $A$ is a field. For $M$ a projective $(\varphi, \Gamma)$-module of rank 1, the \emph{degree} of $M$, denoted $M$, is ...
Note that 
\begin{equation} \label{eq:tensor product degree rank 1}
\deg(M_1 \otimes M_2) = \deg(M_1) + \deg(M_2) \qquad (\rank(M_1) = \rank(M_2) = 1).
\end{equation}
For $M$ a projective $(\varphi, \Gamma)$-module of arbitrary rank, we set
$\deg(M) = \deg(\wedge^{\rank(M)} M)$ according to the previous definition. If $\rank(M) \neq 0$, the \emph{slope} of $M$ is defined as the ratio $\mu(M) = \deg(M)/\rank(M)$.
The degree is additive in short exact sequences: we have
\begin{equation}
0 \to M_1 \to M \to M_2 \to 0 \Longrightarrow \deg(M) = \deg(M_1) + \deg(M_2).
\end{equation}
by virtue of \eqref{eq:tensor product degree rank 1} and the canonical isomorphism
$\wedge^{\rank(M)} M \cong \wedge^{\rank(M_1)} M_1 \otimes \wedge^{\rank(M_2)} M_2$.

\textbf{extend the definition to the torsion case}
\end{defn}

\textbf{definition of slopes in the general case, as functions on $\Maxspec(A)$}

\begin{defn}
\end{defn}

\textbf{\'etale condition, both over a point and not}

\begin{defn}
Suppose that $A$ is a field. \textbf{define the slopes, filtration in that case}.

\textbf{now in the general case, talk about the slopes as a function on $\Maxspec(A)$, and remark that they cannot be defined for Berkovich points}.
\end{defn}


\begin{lemma} \label{L:bounded slopes}
Let $M$ be a projective $(\varphi, \Gamma)$-module. Then as $\frakm$ varies over $\Maxspec(A)$, the slopes of $M/\frakm$ are uniformly bounded above and below.
\end{lemma}
\begin{proof}
...
\end{proof}
 
\begin{remark}
If $A$ is not a field, then the subcategory of $\PhiGamma_{K,A}$ consisting of \'etale objects may be strictly larger than the essential image of $\Rep_A(G_K)$; in fact, this already occurs for objects of rank 1, as noted in \cite[Remarque 4.2.10]{berger-colmez}.
For further discussion, see \cite{kedlaya-liu-families}.
\end{remark}

\section{Cohomology of $(\varphi, \Gamma)$-modules}

\textbf{also talk about cohomology, including the short proof of finiteness?}

\begin{theorem}
coherence, base change
\end{theorem}

\section{Iwasawa cohomology and the cyclotomic deformation}

The goal of this section is to describe various constructions in the classical language of $(\varphi, \Gamma)$-modules which play a role in Iwasawa theory, then translate these into the other categories so as to isolate the role of the cyclotomic extension.

\begin{defn}
Define the map $\psi: \calR_K \to \calR_K$ as the reduced trace of $\varphi$; by definition, it is a left inverse of $\varphi$. For any $(\varphi, \Gamma)$-module $M$,
we may likewise take the reduced trace of the action of $\varphi$ on $M$ to obtain an action of $\psi$ on $M$, which is again a left inverse of $\varphi$. Consequently, we obtain an exact sequence
\begin{equation} \label{eq:psi sequence}
0 \to M^{\varphi=1} \to M^{\psi=1} \stackrel{\varphi-1}{\longrightarrow} M^{\psi=0}.
\end{equation}
The pairing $\{, \}$ satisfies
\begin{equation} \label{eq:pairing phi psi}
\{\varphi(x), y\} = \{x, \psi(y)\} \qquad (x \in M, y \in M^*).
\end{equation}
\end{defn}

The following is \cite[Proposition~3.3.2(1)]{kpx}.
\begin{prop} \label{P:psi finite}
For any $(\varphi, \Gamma)$-module $M$,
the $A$-module $M/(\psi-1)$ is finitely generated.
\end{prop}
\begin{cor}
The pairing $\{-,-\}: M \times M^* \to A$ induces an isomorphism
\[
M^{\varphi=1} \cong \Hom_A(M^*/(\psi-1), A).
\]
In particular, $M^{\varphi=1}$ is a finite $A$-module.
\end{cor}
\begin{proof}
The nondegeneracy of the pairing defines an injection $M \to \Hom_A(M^*, A)$;
the identity \eqref{eq:pairing phi psi} shows that the image of this map is contained in
$\Hom_A(M^*/(\psi-1), A)$. In the other direction, note that Proposition~\ref{P:psi finite} and the open mapping theorem imply that $(\psi-1)M^*$ is a closed subspace of $M^*$
\textbf{for what topology??}
\end{proof}

\begin{prop}[after Chenevier]
Let $M$ be a $(\varphi, \Gamma)$-module.
For any $\gamma \in \Gamma_K$ of infinite order, the map $\gamma-1$ on $M^{\psi=0}$ is bijective.
\end{prop}

One of our goals is to express the terms of \eqref{eq:psi sequence} in terms of constructions that translate to other descriptions of the category of $(\varphi, \Gamma)$-modules. These translations must remove any mention of $\psi$, as this operator has no analogue in the other descriptions.

\begin{prop}
define psi-cohomology and equivalence as in \cite[Definition~2.3.3]{kpx}.
\end{prop}

\section{The cyclotomic deformation}

We define the \emph{cyclotomic deformation} of a $(\varphi, \Gamma)$-module following 
\cite[Definition~4.4.7]{kpx}, but using more geometric language.
\begin{defn}
Let $\ZZ_p \llbracket \Gamma_K \rrbracket$ be the completed group algebra. We may then apply Berthelot's generic fiber construction to view this ring as the collection of bounded-by-1 rigid analytic functions on a certain one-dimensional quasi-Stein space $X$ over $\QQ_p$.
The action of $\ZZ_p \llbracket \Gamma_K \rrbracket$ on $\calO(X)$ by (left) multiplication
defines a a canonical one-dimensional Galois representation on $X$; let $\Dfm$ be the corresponding $(\varphi, \Gamma_K)$-module.

For $M$ a $(\varphi, \Gamma)$-module, 
\end{defn}


The following is \cite[Theorem~4.4.8]{kpx}.
\begin{theorem}
Let $M$ be a $(\varphi, \Gamma_K)$-module over $\calR_{K,A}$. Then...
\end{theorem}


\textbf{compare $M^{\psi=0}$ to the boundary of the cyclotomic deformation}

\appendix

\section{Torsion $(\varphi, \Gamma)$-modules}

As an aside, we resolve a minor open question in the classical theory of $(\varphi, \Gamma)$-modules.

\begin{lemma} \label{L:modification slope bound}
Assume that $A$ is a field. 
Choose $s \in \QQ$.
Let $M$ be a projective $(\varphi, \Gamma)$-module whose slopes are all at least $s$.
Let $T$ be a $t$-torsion $(\varphi, \Gamma)$-module of degree $d$.
Then for any short exact sequence $0 \to N \to M \to T \to 0$,
the slopes of $N$ are all at least $s-d$.
\end{lemma}
\begin{proof}
Let $P$ be the final quotient in the slope filtration of $N$ and put $N' = \ker(N \to P)$.
By pushing out, we form the exact sequence 
\[
0 \to P \to M/N' \to T \to 0.
\]
Put $r = \rank(P)$, so that $\rank(M/N') = r$ and  $\deg(M/N') = r \mu(P)  + d$, yielding  $\mu(M/N') = \mu(P) + d/r$. Since $M/N'$ is a quotient of $M$, we must have
 $\mu(M/N') \geq s$; since $r \in \{1,\dots,n\}$, we deduce the desired result.
\end{proof}



\textbf{define $t$-torsion, $t$-coherent $(\varphi, \Gamma)$-modules}

\begin{lemma} \label{L:lift topological action}
Let $R$ be a Banach algebra over $\QQ_p$. Let $\Gamma$ be a pro-$p$-cyclic group topologically generated by $\gamma$ acting continuously on $R$. Let $F$ be a finite free $R$-module with basis $\be_1,\dots,\be_d$. Fix a semilinear action of $\gamma$ on $F$ with the property that the matrix of action $A$ of $\gamma$ on $\be_1,\dots,\be_d$ satisfies $\left| A-1 \right| < 1$.
Then the action of $\gamma$ on $F$ extends to a continuous action of $\Gamma$.
\end{lemma}
\begin{proof}
Choose $c$ such that $\left| A-1 \right| < c < 1$.
For $n$ a nonnegative integer, let $A_n$ be the matrix of action of $\gamma^{p^n}$
on $\be_1,\dots,\be_d$. It suffices to prove that for all sufficiently large $n$ we have
$\left| A_n-1 \right| \leq \max\{c^2, p^{-1}c\}$, as then iterating the argument (with $\Gamma$ replaced by the closure of the subgroup generated by $\gamma^{p^n}$ for some large $n$) will show that the action of $\gamma^{\ZZ}$ extends continuously to $\Gamma$.

By hypothesis, the action of $\Gamma$ on $R$ is continuous. Consequently, there exists 
$n_0 \geq 0$ such that for all $n \geq n_0+1$, we have $\left| (\gamma^{p^n}-1)(A) \right| \leq c^2$. For such $n$, modulo elements of $R$ of norm at most $\max\{c^2,p^{-1}c\}$ we have
\[
A_{n} = A \gamma(A) \cdots \gamma^{p^{n}-1}(A) \equiv 1 + \sum_{i=0}^{p^{n}-1} (\gamma^i(A)-1) \equiv 1 + p\sum_{j=0}^{p^{n-1}-1} (\gamma^{j}(A)-1) \equiv 1.
\]
This proves the claim.
\end{proof}

\textbf{warning: the twist in the following lemma is the phi-twist. Need to fix notation.}
\begin{lemma} \label{L:lift torsion action}
Let $M$ be a $t$-torsion $(\varphi, \Gamma)$-module. Then there exists
a projective $(\varphi, \Gamma)$-module $F$ such that for each $n \in \ZZ$, there exists a surjection $F(n) \to M$ of $(\varphi, \Gamma)$-modules.
\end{lemma}
\begin{proof}
Represent $M$ as a finitely generated $\widehat{K}_{\infty} \widehat{\otimes}_{\Qp} A$-module equipped with a continuous semilinear $\Gamma_K$-action.
We first make an argument emulating \cite[Lemma~1.5.2]{kedlaya-liu1}.
Choose module generators $\bv_1,\dots,\bv_d$ of $M$,
let $F_0$ be the free module over $\tilde{\calR}^{[p^{-1/2}, p^{1/2}]}_{K,A}$ on the
generators $\be_1,\dots,\be_d$, and form the surjection $F_0 \to M$ taking $\be_i$ to $\bv_i$. For $\gamma \in \Gamma_K$, there exist $d \times d$ matrices
$B_\gamma, C_\gamma$ over $\tilde{\calR}^{[p^{-1/2}, p^{1/2}]}_{K,A}$ such that
$\gamma(\bv_j) = \sum_i B_{\gamma,ij} \bv_i$, $\gamma^{-1}(\bv_j) = \sum_i \gamma^{-1}(C_{\gamma,ij}) \bv_i$. Equip $F_1 = F_0 \oplus F_0$ with the action of $\gamma$ defined on the standard basis by the block matrix
\[
\begin{pmatrix} B_\gamma & B_\gamma C_\gamma -1 \\ 1 & C_\gamma
\end{pmatrix};
\]
then this action of $\gamma$ is invertible and the map $F_1 \to M$ given by projecting onto the first copy of $F_0$ and then onto $M$ is $\gamma$-equivariant.

Recall that by hypothesis, the action of $\Gamma_K$ on $M$ is continuous. Consequently,
for $\gamma$ sufficiently small, the matrices $B_\gamma, C_\gamma$ can be chosen so that
$\left| B_\gamma-1 \right|, \left| C_\gamma -1 \right| < 1$.
By Lemma~\ref{L:lift topological action}, the action of $\gamma$ on $F_1$ extends to an open subgroup $\Gamma'$ of $\Gamma_K$, and the map $F_1 \to M$ is necessarily also $\Gamma'$-equivariant. Put $F = \Ind_{\Gamma'}^{\Gamma_K} F_1$; then $F_1 \to M$ corresponds to a $\Gamma$-equivariant surjection $F \to M$. 

We may choose any matrix of action for $\varphi$ (with coefficients in $\tilde{\calR}^{[p^{1/2}, p^{1/2}]}_{K,A}$)
to promote $F$ to a projective $(\varphi, \Gamma)$-module equipped with an equivariant surjection $F \to M$. Given one such choice, we may then multiply the matrix of action by powers of $p$ to obtain maps $F(n) \to M$ for $n \in \ZZ$. 
\end{proof}

\begin{theorem} \label{T:t-coherent lift}
Every $t$-coherent $(\varphi, \Gamma)$-module admits a surjective morphism from some projective $(\varphi, \Gamma)$-module.
\end{theorem}
\begin{proof}
Let $M$ be a $t$-coherent $(\varphi, \Gamma)$-module. Let $T$ be the $t$-torsion submodule of $M$; by hypothesis, $F = M/T$ is a projective $(\varphi, \Gamma)$-module.
By Lemma~\ref{L:lift torsion action}, there exists a projective $(\varphi, \Gamma)$-module $F'$ such that  for each $n \in \ZZ$, there exists a surjection $F'(n) \to T$ of $(\varphi, \Gamma)$-modules. Let $K_n$ be the kernel of one such surjection. 

We claim that for $n$ sufficiently large, $H^2_{\varphi, \Gamma}(F^\dual \otimes K_n) = 0$.
To check this, first apply Lemma~\ref{L:bounded slopes} twice to produce values $s_0, s_1$ such that $F$ has all slopes at most $s_0$ while $F'$ has all slopes at least $s_1$.
Then apply Lemma~\ref{L:modification slope bound} to see that for any
$n \geq s_0-s_1 + \deg(T)$,
$F^\dual \otimes K_n$ has everywhere positive slopes. 
By \textbf{coherence}, for such $n$ we have $H^2_{\varphi, \Gamma}(F^\dual \otimes K_n) = 0$; consequently, $H^1_{\varphi, \Gamma}(F^\dual \otimes F'(n)) \to H^1_{\varphi, \Gamma}(F^\dual \otimes T)$ is surjective. In particular, there exists a commutative diagram
\[
\xymatrix{
0 \ar[r] & F'(n) \ar[r] \ar[d] & M' \ar[r] \ar[d] & F \ar[r] \ar@{=}[d] & 0 \\
0 \ar[r] & T \ar[r] & M \ar[r] & F \ar[r] & 0
}
\]
with exact rows; by the five lemma, $M' \to M$ is surjective, proving the claim.
\end{proof}

\begin{remark}
Theorem~\ref{T:t-coherent lift} resolves a question raised implicitly in
\cite{kedlaya-bordeaux} by showing that in the terminology of that paper,
every generalized $(\varphi, \Gamma)$-module is a $B$-quotient.
\end{remark}

\begin{remark}
One of the more delicate parts of \cite{kpx} is the proof of \textbf{coherence}. 
We would like to indicate here a simpler proof using the Cartan-Serre method (i.e., the Schwartz lemma).

...
\end{remark}



\begin{thebibliography}{99}

\bibitem{beauville-laszlo}
A. Beauville and Y. Laszlo, Un lemme de descente, \textit{C. R. Acad. Sci. Paris} \textbf{320} (1995), 335--340.

\bibitem{berger-colmez}
L. Berger and P. Colmez, Familles de repr\'esentations de de Rham et monodromie $p$-adique, \textit{Ast\'erisque} \textbf{319} (2008), 303--337.

\bibitem{chenevier}
G. Chenevier, Sur la densit\'e des repr\'esentations cristallines de
$\Gal(\overline{\QQ}_p/\QQ_p)$, \textit{Math. Ann.} \textbf{355} (2013), 1469--1525.

\bibitem{kedlaya-bordeaux}
K.S. Kedlaya, Some new directions in $p$-adic Hodge theory, \textit{J. Th\'eorie Nombres Bordeaux} \textbf{21} (2009), 285--300.

\bibitem{kedlaya-new-phigamma}
K.S. Kedlaya, New methods for $(\varphi, \Gamma)$-modules,
to appear in \textit{Res. Math. Sci.}

\bibitem{kedlaya-noetherian}
K.S. Kedlaya, Noetherian properties of Fargues-Fontaine curves,
\textit{Int. Math. Res. Notices} (2015), article ID rnv227. 

\bibitem{kedlaya-liu-families}
K.S. Kedlaya and R. Liu, On families of $(\varphi, \Gamma)$-modules,
\textit{Alg. and Number Theory} \textbf{4} (2010), 943--967.

\bibitem{kedlaya-liu1}
K.S. Kedlaya and R. Liu, Relative $p$-adic Hodge theory: Foundations,
\textit{Ast\'erisque} \textbf{371} (2015), 239 pages. 

\bibitem{kpx}
K.S. Kedlaya, J. Pottharst, and L. Xiao, Cohomology of arithmetic families of $(\varphi, \Gamma)$-modules, \textit{J. Amer. Math. Soc.}  \textbf{27} (2014), 1043--1115. 

\end{thebibliography}

\end{document}
