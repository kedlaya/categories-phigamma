\documentclass[12pt]{amsart}
\usepackage{fullpage}

\begin{document}

\begin{center}
Changes from version 1 to version 2
\end{center}

The introduction has been completely rewritten, with the goals of providing more historical motivation, indicating how the paper fits into the context of recent work in the area, and providing more side pointers into the literature for topics that may be of related interest.

Definition 1.2: Direct limits of topological spaces are a bit touchy; we edited the text slightly to clarify.

Definition 1.3: We point out that the Robba ring is connected whenever $A$ is. This means that the rank function in Definition 1.4 factors through $\mathrm{Spec}\, A$.

Example 2.2: The way the example written does in fact presume $K = F$. To cover the general case, one should appeal to Theorem 2.4.

Definition 2.5: Some remarks were added at the end to address the question about ``power series in $p$.''

Lemma 2.6: For essential surjectivity, ``tensoring`` with the splitting does not work, but there is a more sophisticated argument using the splitting that does give descent, based on a formal result of Joyal--Tierney.

Definition 2.8: The bijectivity for $\varphi$-modules comes from bijectivity on the underlying ring. We aren't sure whether any further comment is warranted.

Definition 2.17: We don't know of any significance to $Z_{L,A}$ being contained in an open affine subscheme besides the fact that it implies that $\widehat{Z}_{L,A}$ is affine.

Remark 2.19: We added some further discussion about the interplay between the schematic and adic Fargues--Fontaine curves, in order to illustrate the ``meaning'' of the spaces $P_{L,A,n}$ and how they relate to the extended Robba rings.

Remark 2.20, Lemma 2.21: These have been inserted to summarize the contact between perfectoid fields and older notions of ``deep ramification'' in the theory of local fields.

Definition 3.3: The valuation map is perhaps not most easily understood in terms of representations in tensors; we give a slightly different exposition. Similarly, while the degree map can indeed be constructed in terms of rational sections (this being the point of view taken in Fargues--Fontaine), it is easier in this context to express it in terms of $\varphi$-modules.

Lemma 3.4: We fleshed out slightly the ``induction'' comment.

Remark 4.9: This has been added to indicate the extent to which one can replace the cyclotomic tower, thus motivating the subsequent discussion.

Lemma 5.2: To alleviate confusion, we rewrote the statement to make it more clear what morphisms are giving rise to the quasi-isomorphisms in question.

Remark 5.8: We describe the analogue of Shapiro's lemma in more detail.

Remark 5.9: This remark has been added to clarify what happens in Theorem 5.7 when $K$ is not required to be finite over $\mathbb{Q}_p$.

Definition 6.1: Justification for the sheaf axiom (and acyclicity) added.

Definition 6.2: The group algebra is indeed formally of finite type over $\mathbb{Z}_p$. We added the standard reference for Berthelot's generic fiber construction (de Jong's paper on crystalline Dieudonn\'e theory), as well as some further explanation of the shape of $W_K$. The canonical representation is one-dimensional simply because it is acting on a one-dimensional projective module over $\mathcal{O}(W_K)$. Regarding the comment ``It would be nice to explain what this looks like in the case when $X$ itself is the generic rigid fiber of some formal scheme...'', it is not clear to us how to give such a description; it would depend on the module $M$, which is typically not easy to describe in terms of formal schemes.

Remark 7.4: To answer a referee's comment, we added some discussion of what $D_{\mathrm{crys}}$ means for a $(\varphi, \Gamma)$-module.


\end{document}