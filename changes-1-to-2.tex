\documentclass[12pt]{amsart}
\usepackage{fullpage}

\begin{document}

\begin{center}
Changes from version 1 to version 2
\end{center}

Lemma 5.2: To alleviate confusion, we rewrote the statement to make it more clear what morphisms are giving rise to the quasi-isomorphisms in question.

Remark 5.8: We describe the analogue of Shapiro's lemma in more detail.

Remark 5.9: This remark has been added to clarify what happens in Theorem 5.7 when $K$ is not required to be finite over $\mathbb{Q}_p$.

Definition 6.1: Justification for the sheaf axiom (and acyclicity) added.

Definition 6.2: The group algebra is indeed formally of finite type over $\mathbb{Z}_p$. We added the standard reference for Berthelot's generic fiber construction (de Jong's paper on crystalline Dieudonn\'e theory), as well as some further explanation of the shape of $W_K$. The canonical representation is one-dimensional simply because it is acting on a one-dimensional projective module over $\mathcal{O}(W_K)$. Regarding the comment ``It would be nice to explain what this looks like in the case when $X$ itself is the generic rigid fiber of some formal scheme...'', it is not clear to us how to give such a description; it would depend on the module $M$, which is typically not easy to describe in terms of formal schemes.

Remark 7.4: To answer a referee's comment, we added some discussion of what $D_{\mathrm{crys}}$ means for a $(\varphi, \Gamma)$-module.

Section 8: We inserted Remark 8.1 and Lemma 8.2 to summarize the contact between perfectoid fields and older notions of ``deep ramification'' in the theory of local fields.

\end{document}